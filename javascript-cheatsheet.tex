\documentclass{article}
\title{Javascript Cheatsheet}
\author{Puvendran Pillay}
% Taken from Lena Herrmann at
% http://lenaherrmann.net/2010/05/20/javascript-syntax-highlighting-in-the-latex-listings-package
\usepackage{listings}
\lstdefinelanguage{JavaScript}{
  keywords={typeof, new, true, false, catch, function, return, null, catch, switch, var, if, in, while, do, else, case, break},
  keywordstyle=\bfseries,
  ndkeywords={class, export, boolean, throw, implements, import, this},
  ndkeywordstyle=\bfseries,
  %identifierstyle=,
  sensitive=false,
  comment=[l]{//},
  morecomment=[s]{/*}{*/},
  commentstyle=\ttfamily,
  stringstyle=\ttfamily,
  morestring=[b]',
  morestring=[b]"
}

\lstset{
   language=JavaScript,
   %backgroundcolor=,
   extendedchars=true,
   basicstyle=\footnotesize\ttfamily,
   showstringspaces=false,
   showspaces=false,
   numbers=left,
   numberstyle=\footnotesize,
   numbersep=9pt,
   tabsize=2,
   breaklines=true,
   showtabs=false,
   captionpos=b
}

\begin{document}
  \maketitle
  \section{Comments}
    \begin{lstlisting}
      // in-line comment
      /* multi-line
      comment */
    \end{lstlisting}
  \section{Datatypes}
    \begin{enumerate}
      \item undefined
      \item  null
      \item  boolean
      \item  string
      \item  number
      \item  object
      \item  array
      \item  float
    \end{enumerate}
  \section{Variables}
    \begin{enumerate}
      \item Declare variable
        \begin{lstlisting}
          var ourName;
        \end{lstlisting}
      \item Assign value to a variable
        \begin{lstlisting}
          ourName = "something";
        \end{lstlisting}
      \item Can also be done once
        \begin{lstlisting}
          var ourName = "something";
        \end{lstlisting}
        \item Variable should be initialized when declared.
        \item Javascript variables names are \emph{case-sensitive}.
    \end{enumerate}
  \section{Operators}
      \begin{enumerate}
        \item Arithmetic Operators
          \begin{enumerate}
            \item Addition
              \begin{lstlisting}
                var x = 5;     // assign the value 5 to x
                var y = 2;     // assign the value 2 to y
                var z = x + y  // assign the value 7 to z (x + y)
                alert(z);      // outputs the value in an alert box
              \end{lstlisting}
            \item Subtraction
              \begin{lstlisting}
                var x = 5;     // assign the value 5 to x
                var y = 2;     // assign the value 2 to y
                var z = x - y  // assign the value 3 to z (x - y)
                alert(z);      // outputs the value in an alert box
              \end{lstlisting}
            \item Multiplication
              \begin{lstlisting}
                var x = 5;     // assign the value 5 to x
                var y = 2;     // assign the value 2 to y
                var z = x * y  // assign the value 10 to z (x * y)
                alert(z);      // outputs the value in an alert box
              \end{lstlisting}
            \item Division
              \begin{lstlisting}
                var x = 5;     // assign the value 5 to x
                var y = 2;     // assign the value 2 to y
                var z = x / y  // assign the value 2.5 to z (x / y)
                alert(z);      // outputs the value in an alert box
              \end{lstlisting}
            \item Modulus \textit{(returns the remainder of division)}
              \begin{lstlisting}
                var x = 5;     // assign the value 5 to x
                var y = 2;     // assign the value 2 to y
                var z = x % y  // assign the value 1 to z (x % y)
                alert(z);      // outputs the value in an alert box
              \end{lstlisting}
            \item Increment
              \begin{lstlisting}
              //same as var x = x + 1;
              var x = 0;       // assigns the value of 0 to x
              x++;             // increases the value by 1
              alert(x);        // outputs the value in an alert box
              \end{lstlisting}
            \item Decrement
              \begin{lstlisting}
              //same as var x = x - 1;
              var x = 0;       // assigns the value of 0 to x
              x--;             // Decreases the value by 1
              alert(x);        // outputs the value in an alert box
              \end{lstlisting}
          \end{enumerate}
        \item Assignment Operators
        \begin{enumerate}
          \item assignment (=)
            \begin{lstlisting}
              var x = 10;
              alert(x);  // outputs the value in an alert box
            \end{lstlisting}
          \item Addition assignment
            \begin{lstlisting}
              var x = 10;
              x += 5;    // same as x = x + 5;
              alert(x);  // outputs the value in an alert box
            \end{lstlisting}
          \item Subtraction assignment
            \begin{lstlisting}
              var x = 10;
              x -= 5;    // same as x = x - 5;
              alert(x);  // outputs the value in an alert box
            \end{lstlisting}
          \item Multiplication assignment
            \begin{lstlisting}
              var x = 10;
              x *= 5;    // same as x = x * 5;
              alert(x);  // outputs the value in an alert box
            \end{lstlisting}
          \item Division assignment
            \begin{lstlisting}
              var x = 10;
              x /= 5;    // same as x = x / 5;
              alert(x);  // outputs the value in an alert box
            \end{lstlisting}
          \item Modulus assignment
            \begin{lstlisting}
              var x = 10;
              x %= 5;    // same as x = x % 5;
              alert(x);  // outputs the value in an alert box
            \end{lstlisting}
        \end{enumerate}
      \end{enumerate}
  \section{Escape Characters}
    \begin{enumerate}
      \item Single quote (\textbackslash')
      \item Double quote (\textbackslash")
      \item Backsplash (\textbackslash\textbackslash)
      \item New line (\textbackslash n)
      \item Carriage return (\textbackslash r)
      \item Tab (\textbackslash t)
      \item Backspace (\textbackslash b)
      \item Form Feed (\textbackslash f)
      \item Examples
        \begin{lstlisting}
          var x = 'It\'s alright';
          var y = "We are the so-called \"Vikings\" from the north."
        \end{lstlisting}
    \end{enumerate}
  \section{Concatenation}
      \begin{enumerate}
        \item Concatenate String with another string
          \begin{lstlisting}
            var x = "Hello";
            var y = "World!";
            alert(x + " " + y); // outputs x and y with space in the middle
          \end{lstlisting}
        \item Concatenate String with a variable
          \begin{lstlisting}
            var x = "Hello";
            x += "World!";  // attaches the word to x
            alert(x);
          \end{lstlisting}
      \end{enumerate}
  \section{".length" Property}
    Used to find the total number of characters contained
    \begin{lstlisting}
      var x ="Hello World!";
      alert(x.length);
    \end{lstlisting}
  \section{"[]"}
    Used to index character \emph{(Javascript starts counting from 0 not 1)}
      \begin{lstlisting}
        var x ="Hello World!";
        alert(x[1]); //selects "e" from "Hello"
      \end{lstlisting}
    Strings are immutable that means you can't change individual characters in them.
      \begin{lstlisting}
        var x = "Jello World";
        x[1] ="H"; //it just doesn't work
        x = "Hello World";  // you must change it completely
      \end{lstlisting}
  \section{Arrays}
    \begin{enumerate}
      \item One dimensional array
        \begin{lstlisting}
          var x = ["Hello" , "World"];
        \end{lstlisting}
      \item Multi-dimensional array
        \begin{lstlisting}
          var x =[["Hello" , "World"] , ["Good" , "Bye"]];
        \end{lstlisting}
      \item Arrays are mutable and data can be changed individually
      \item "[]" are also used to index arrays
        \begin{lstlisting}
          var x =[["Hello" , "World"] , ["Good" , "Bye"]];
          alert(x[1][0]); // outputs "Good"
        \end{lstlisting}
      \item ".push()" property
        Used to append data \emph{to "push in from the back"}
          \begin{lstlisting}
            var x = ["Hello" , "World"];
            x.push(["Nasi" , "Lemak"]);
            alert(x);
          \end{lstlisting}
      \item ".pop()" property
        Used to unappend data \emph{to "pop out the back"}
          \begin{lstlisting}
            var x = ["Hello" , "World"];
            x.push();
            alert(x);
          \end{lstlisting}
      \item ".shift()" property
        Used to unprepend data \emph{to "shift out the front"}
          \begin{lstlisting}
            var x = ["Hello" , "World"];
            x.push();
            alert(x);
          \end{lstlisting}
      \item ".unshift()" property
        Used to prepend data \emph{to "unshift the front"}
          \begin{lstlisting}
            var x = ["Hello" , "World"];
            x.unshift(["Nasi" , "Lemak"]);
            alert(x);
          \end{lstlisting}
    \end{enumerate}
  \section{Functions}
    \begin{enumerate}
      \item Use the keyword "function" to create a function
        \begin{lstlisting}
          function favFood(){
            alert("I like eating my favourite food");
          }
        \end{lstlisting}
      \item Functions can have arguments
          \begin{lstlisting}
            function favFood(name , food){
              var name = prompt("What is your name?");
              var food = prompt("What is your favourite food?");
              alert( name + " " + "like eating" + " " + food);
            }
          \end{lstlisting}
      \item Scopes
        Scope is the visiblity of variables
        \begin{enumerate}
          \item Local scope \emph{(only available within functions)}
            \begin{lstlisting}
              function favFood(name , food){
                var name = prompt("What is your name?");  // Local scope!
                var food = prompt("What is your favourite food?");  // Local scope!
                alert( name + " " + "like eating" + " " + food );
              }
              alert( name + " " + "likes" + " " + food );
            \end{lstlisting}
          \item Global scope \emph{(available everywhere)}
            \begin{lstlisting}
              var name; // declare outside
              var food; // the function to make it global

              function favFood(name , food){
                name = prompt("What is your name?");
                food = prompt("What is your favourite food?");
                alert( name + " " + "like eating" + " " + food );
              }
              alert( name + " " + "likes"+ " " + food );  // can be used again
            \end{lstlisting}
          \item Local scope takes predence over Global scope
        \end{enumerate}
      \item "return" statement
      The return statement stops the execution of a function and returns a value from that function.
        \begin{lstlisting}
          function helloWorld(){
            return 1 + 1;
          }
        \end{lstlisting}
    \end{enumerate}
  \section{Boolean}
  Boolean are either True or False
  \section{Conditional Statements}
  Used to specify a block of code to execute, if conditions are true
    \begin{enumerate}
      \item If...elseif...else Statements
      \begin{lstlisting}
        var age = 20;
        // this condition is false, so the code moves to the next one
        if (age < 17){
        alert("You can't drive!");
        } elseif (age > 17){
        alert("You can drive!");
        }else{
        alert("You should drive!");
        }
      \end{lstlisting}
      \item Comparison and Logical Operators
      \begin{enumerate}
        \item Equal (==)
        \item Strict Equal (===)
        \item Not Equal (!=)
        \item Not Strict Equal (!==)
        \item Greater than (\textgreater)
        \item Greater than or Equal to (\textgreater=)
        \item Less than (\textless)
        \item Less than or Equal to (\textless=)
        \item AND operator ({\&}{\&})
        \item OR operator (\textbar\textbar)
      \end{enumerate}
      \item Switch Statements
      The switch expression is evaluated once. The value of the expression is compared with the values of each case. If there is a match, the associated block of code is executed.
        \begin{lstlisting}
          var age = 20;
          switch (age){
          case 16:
            alert("You can't drive!");
            break;
          case 17:
          case 18:
          case 19:
          case 20:
            alert("You can drive!");
            break;
          default:
            alert("You should drive!");
          }
        \end{lstlisting}
    \end{enumerate}
  \section{Objects}
  Objects contains more than one value
    \begin{enumerate}
      \item Example
        \begin{lstlisting}
          var dog ={
            "name":"Clifford",
            "legs": 4,
            "goodBoy": TRUE,
            };
        \end{lstlisting}
      \item Methods
      Used to change data in a object
        \begin{enumerate}
          \item (.) or ([]) to access properties in an object
            \begin{lstlisting}
              var dog ={
                "name":"Clifford",
                "legs": 4,
                "goodBoy": TRUE,
                };
                alert(dog.name);
            \end{lstlisting}
          \item ".delete" method
          to remove the properties
          \item "hasOwnProperty()" method
          to check whether property exists
        \end{enumerate}
    \end{enumerate}
  \section{"for" Loop}
  Loops can execute a block of code a number of times
    \begin{lstlisting}
      function counter(){
        var count="";
        var i;
        for (i = 0; i < 5; i++){
        count += "Counting" + i;}
        console.log(count);
      }
      counter();
    \end{lstlisting}
  \section{Math Function}
    \begin{enumerate}
      \item "Math.random()" function
      to generate a random number between 0 and 1
      \item "Math.floor()" function
      to round down a number to the nearest integer
    \end{enumerate}
  \section{Regular Expression}
  A regular expression is a sequence of characters that forms a search pattern. The search pattern can be used for text search and text replace operations.
    \begin{enumerate}
      \item Example
        \begin{lstlisting}
          var text = "this is a text"
          var i =/this/gi //search term in between two //
          var thisCount = testString.match(i).length;
          alert(thisCount);
        \end{lstlisting}
      \item "g" expression \emph{(global)}
      to find all matches
      \item "i" expression
      to ignore case-sensitivity
      \item  "\textbackslash d"
      to retrieve digits
      \item "\textbackslash s"
      to find whitespace
      \item  "\textbackslash S"
      to find non-whitespace
      \item {( \textbackslash d+)}
      to get multiple results
    \end{enumerate}
  \section{Constructor functions}
  to add properties in an object
  \begin{lstlisting}
    var Car = function() {
      //"this" refers to the new object being created by the constructor
      this.wheels = 4;
      this.engines = 1;
      this.seats = 1;
    };
    //"new" keyword to call constructor
    // name of contructor is CAPITALIZED
    var myCar =  new Car(); {
      this.wheels = 4;
      myCar.nickname = "Vroom";
    }
  \end{lstlisting}
\end{document}
